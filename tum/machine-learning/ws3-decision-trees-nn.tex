
\documentclass{article}
\usepackage{ws_template}
\usepackage{amsmath}
\usepackage{amssymb}
\usepackage{a4wide}
\usepackage{graphicx}
\usepackage{booktabs}

% please submit the corresponding pdf by email to
% homework@class,brml.org, and write "homework sheet xx" in the 
% title.  No more, no less!  (Instead of xx, however,
% put the decimal number of the homework sheet.)

% Please update the following line, only change XX to the homework
% sheet number
\title{homework sheet 03}


\author{
\name{Denys Sobchyshak}\\
\imat{03636581}\\
\email{denys.sobchyshak@gmail.com}
\And
\name{Sergey Zakharov} \\
\imat{03636642}\\
\email{ga39pad@mytum.de}
}

% The \author macro works with any number of authors. There are two commands
% used to separate the names and addresses of multiple authors: \And and \AND.
%
% Using \And between authors leaves it to \LaTeX{} to determine where to break
% the lines. Using \AND forces a linebreak at that point. So, if \LaTeX{}
% puts 3 of 4 authors names on the first line, and the last on the second
% line, try using \AND instead of \And before the third author name.


\begin{document}
\maketitle

\section{Learning by doing}

\textbf{Problem 1:}\\

Root node: $ Gini(5,6,4) = 1 - (\frac{5}{15})^2 - (\frac{6}{15})^2 - (\frac{4}{15})^2 \approx 0.658 $ \\

First split: $x_1 < 4.2$:\\

Left node: $Gini(0,6,0) = 1 - (\frac{6}{6})^2 = 0 = Cost_L $\\
Right node: $Gini(5,0,4) = 1 - (\frac{5}{9})^2 - (\frac{4}{9})^2 \approx 0.494; \quad  Cost_{R} = \frac{9}{15} * 0.494 \approx 0.296 $\\

Second split: $x_1 < 7$:\\

Left subnode: $Gini(2,0,4) = 1 - (\frac{2}{6})^2 - (\frac{4}{6})^2 \approx 0.444; \quad Cost_{RL} = \frac{6}{9} * 0.444 \approx 0.296 $\\
Right subnode: $Gini(3,0,0) = 1 - (\frac{3}{3})^2 = 0 = Cost_{RR} $\\

\textbf{Problem 2:}\\

\[ p(c = 1 | x_a, T) = 1\]
\[ p(c = 2 | x_b, T) = \frac{2}{3}\]

\textbf{Problem 3:}\\

The nearest points to the vector $ x_a $ are I, C and O with the distances 0.67, 2.18 and 2.47 respectively. Unfortunately all the neighbours are of different classes, which produces a tie. The class can therefore be assigned randomly for K = 3.

The nearest points to the vector $ x_b $ are E, I and C with the distances 1.17, 1.75 and 2.12 respectively. 2 out of 3 point are of the class 2. So, $ p(c = 2 | x_b, K = 3) = \frac{2}{3} $
\newpage
\textbf{Problem 4:}\\

\[ \hat{z}_a = \frac{1}{C}\sum_{i \in N_3(x_a)} { \frac{z_i}{d(x_a,x_i)}} = \frac{\frac{2}{2.18} + \frac{1}{2.47}}{\frac{1}{0.67} + \frac{1}{2.18} + \frac{1}{2.47}} \approx 0.56   \]
\[ \hat{z}_b = \frac{1}{C}\sum_{i \in N_3(x_b)} { \frac{z_i}{d(x_b,x_i)}} = \frac{\frac{2}{1.17} + \frac{2}{2.12}}{\frac{1}{1.17} + \frac{1}{1.75} + \frac{1}{2.12}} \approx 1.39 \]

\textbf{Problem 5:}\\

The problem is in the second feature, namely it has a very low standard deviation comparing with the other features. This causes an information lost, because of the little influence of the second feature on the overall result. A possible solution would be achieved by rescaling all the features by, for example, using the Mahalanobis distance.

For the decision trees such a problem doesn't appear as they are scale invariant.

\end{document}
